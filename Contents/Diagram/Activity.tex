\subsubsection{Quản lý tài khoản}
\subsubsubsection{Đăng ký}
\begin{figure}[H]
    \centering
     \includesvg[width=1\textwidth]{Dg_Activity/ManageAccount_Signup.svg}
    \vspace{0.5cm}
    \caption{Biểu đồ activity cho Đăng ký}
    \label{fig:enter-label}
\end{figure}
\subsubsubsection{Đăng nhập}
\begin{figure}[H]
    \centering
     \includesvg[width=1\textwidth]{Dg_Activity/ManageAccount_Login.svg}
    \vspace{0.5cm}
    \caption{Biểu đồ activity cho Đăng nhập}
    \label{fig:enter-label}
\end{figure}
\subsubsubsection{Chỉnh sửa hồ sơ cá nhân}
\begin{figure}[H]
    \centering
     \includesvg[width=1\textwidth]{Dg_Activity/ManageAccount_Profile.svg}
    \vspace{0.5cm}
    \caption{Biểu đồ activity cho Chỉnh sửa hồ sơ cá nhân}
    \label{fig:enter-label}
\end{figure}


\subsubsection{Tích hợp}
\begin{figure}[H]
    \centering
     \includesvg[width=1\textwidth]{Dg_Activity/Activity Diagram-Integration.drawio.svg}
    \vspace{0.5cm}
    \caption{Biểu đồ activity cho Tích hợp}
    \label{fig:enter-label}
\end{figure}

\par Quá trình tạo mã nhúng cho chatbot bắt đầu bằng việc người dùng điều hướng tới trang Tùy chỉnh chatbot. Sau khi truy cập thành công, hệ thống sẽ hiển thị giao diện Tùy chỉnh chatbot và cho phép xem trước giao diện chatbot để đảm bảo rằng thiết kế và các chức năng hoạt động như mong muốn trước khi triển khai. Khi đã hài lòng với bản xem trước, người dùng sẽ nhấn nút Tạo chatbot để hệ thống bắt đầu quá trình tạo mã nhúng. Sau khi hoàn thành, mã nhúng này sau đó sẽ được hiển thị trên giao diện người dùng, cho phép người dùng dễ dàng thao tác. Cuối cùng, người dùng chỉ cần sao chép mã nhúng và dán vào website doanh nghiệp của mình để hoàn tất quá trình tích hợp chatbot.

\subsubsection{Quản lý chatbot}
\subsubsubsection{Báo cáo}
\begin{figure}[H]
    \centering
     \includesvg[width=1\textwidth]{Dg_Activity/Activity Diagram-Report.drawio.svg}
    \vspace{0.5cm}
    \caption{Biểu đồ activity cho Báo cáo}
    \label{fig:enter-label}
\end{figure}
\subsubsubsection{Lịch sử hội thoại}
\begin{figure}[H]
    \centering
     \includesvg[width=1\textwidth]{Dg_Activity/Activity Diagram-Chat.drawio.svg}
    \vspace{0.5cm}
    \caption{Biểu đồ activity cho Lịch sử hội thoại}
    \label{fig:enter-label}
\end{figure}

\subsubsection{Thông báo và cảnh báo}
\begin{figure}[H]
    \centering
     \includesvg[width=1\textwidth]{Dg_Activity/Activity Diagram-Notifications and Alerts.drawio.svg}
    \vspace{0.5cm}
    \caption{Biểu đồ activity cho Thông báo và cảnh báo}
    \label{fig:enter-label}
\end{figure}

\par Quá trình cài đặt thông báo cho người dùng bắt đầu bằng việc người dùng điều hướng tới trang cài đặt thông báo. Khi truy cập thành công, hệ thống sẽ hiển thị trang cài đặt thông báo, cung cấp giao diện để người dùng tùy chỉnh các thiết lập phù hợp với nhu cầu cá nhân.

Tiếp theo, người dùng chọn phương thức nhận thông báo mà họ mong muốn, chẳng hạn như qua email, SMS, hoặc thông báo trực tiếp trên ứng dụng. Sau đó, người dùng thiết lập ngưỡng token LLM hoặc ngưỡng lưu lượng người dùng, nhằm xác định các điều kiện kích hoạt thông báo phù hợp với hoạt động của họ.

Sau khi hoàn tất việc thiết lập các thông số trên, người dùng xác nhận các cài đặt. Hệ thống sẽ áp dụng các cài đặt đã được xác nhận, đảm bảo rằng các thông báo sẽ được gửi đi theo các tùy chọn và ngưỡng đã được thiết lập.

\subsubsection{Quản lý thanh toán}
\begin{figure}[H]
    \centering
     \includesvg[width=1\textwidth]{Dg_Activity/Activity Diagram-Subscription.drawio.svg}
    \vspace{0.5cm}
    \caption{Biểu đồ activity cho Quản lý thanh toán}
    \label{fig:enter-label}
\end{figure}

\par Quá trình đăng ký gói dịch vụ bắt đầu bằng việc người dùng điều hướng tới trang thanh toán. Khi truy cập thành công, hệ thống sẽ hiển thị các gói đăng ký hiện có, cho phép người dùng lựa chọn phù hợp với nhu cầu của mình.

Tiếp theo, người dùng có thể thay đổi hoặc chọn gói đăng ký mới nếu muốn nâng cấp hoặc điều chỉnh dịch vụ. Sau khi lựa chọn gói phù hợp, hệ thống sẽ hiển thị màn hình xác nhận, yêu cầu người dùng xác nhận lựa chọn của mình trước khi tiến hành thanh toán.

Sau khi người dùng xác nhận gói đăng ký, hệ thống sẽ điều hướng tới hệ thống thanh toán để thực hiện giao dịch. Giao diện thanh toán sẽ được hiển thị, nơi người dùng nhập thông tin thanh toán cần thiết như thông tin cá nhân, thông tin thẻ tín dụng hoặc số tài khoản ngân hàng.

Khi người dùng nhập thông tin thanh toán, hệ thống sẽ tiến hành xử lý thanh toán. Nếu thanh toán thành công, hệ thống sẽ kích hoạt gói đăng ký mà người dùng đã chọn, đảm bảo rằng dịch vụ được cung cấp đầy đủ theo gói đã mua. Trong trường hợp thanh toán thất bại, hệ thống sẽ báo lỗi và cho phép người dùng nhập lại thông tin thanh toán hoặc hủy bỏ giao dịch nếu họ quyết định không tiếp tục.


\subsubsection{Tương tác với chatbot}
\subsubsubsection{Phản hồi tin nhắn khách hàng}
\begin{figure}[H]
    \centering
     \includesvg[width=1\textwidth]{Dg_Activity/Activity Diagram-ReplyCustomerMessage.drawio.svg}
    \vspace{0.5cm}
    \caption{Biểu đồ activity cho Phản hồi tin nhắn khách hàng}
    \label{fig:enter-label}
\end{figure}
\subsubsubsection{Gửi tin nhắn}
\begin{figure}[H]
    \centering
     \includesvg[width=1\textwidth]{Dg_Activity/Activity Diagram-SendMessage.drawio.svg}
    \vspace{0.5cm}
    \caption{Biểu đồ activity cho Gửi tin nhắn}
    \label{fig:enter-label}
\end{figure}

\subsubsection{Quản lý lịch hẹn}




